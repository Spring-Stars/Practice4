\documentclass[oneside, a5paper,10pt]{article}
\usepackage{System/core}

\newcommand*\rfrac[2]{{}^{#1}\!/_{#2}}
\renewcommand{\proofname}{Доказательство}

\newcounter{num_authors}
\setcounter{num_authors}{0}

\newarray\authorName
\newarray\authorOrganisation
\newarray\authorMail

%%\isequal{что сравнить}{с чем сравнить}{если да}{если нет}
%%делает проверку, одинаковы ли макросы "что сравнить" и "с чем сравнить". Если одинаковы - выполняет "если ла", в противном случае - "если нет"
%Команда бесполезна.
\newcommand{\isequalmacros}[4]{
\def\a{#1}\def\b{#2}\ifx\a\b#3\else#4\fi
}

\def\ifEqStringBase#1#2%
{\def\csa{#1}\def\csb{#2}\ifx\csa\csb}

%\ifEqString{СТРОКА1}{СТОКА2}{УС_ДА}{УС_НЕТ}
%сравнивает строки СТРОКА1 и СТРОКА 2. Если равны, то выполняет УС_ДА, если нет -- УС_НЕТ.
\newcommand{\ifEqString}[4]{\ignorespaces
\ifEqStringBase{#1}{#2}#3\else#4\fi
}


%%\addAuthor{ФИО}{организация}{e-mail}
%%Добавляет автора
%%Нумерация массива - С ЕДИНИЦЫ!!
\newcommand{\addAuthor}[3]{
\addtocounter{num_authors}{1}
\authorName(\arabic{num_authors})={#1}
\authorOrganisation(\arabic{num_authors})={#2}
\authorMail(\arabic{num_authors})={#3}
}

%\roflll{\x}
\newcommand{\roflll}[1]{
\ignorespaces
\checkauthorMail(#1)\def\mystring{\cachedata}\StrLen{\mystring}[\mystringlenn]\ifnum\mystringlenn>0\checkauthorName(#1)\cachedata$^{*\text{#1}}$\else\checkauthorName(#1)\cachedata$^{\text{#1}}$\fi}

\newcommand{\makeAuthorList}{
\ignorespaces
\begin{center}
\ifthenelse{\arabic{num_authors} = 0}{}{
\foreach\x in{1,...,\arabic{num_authors}}{
\roflll{\x}\ifnum\x<\arabic{num_authors}{,}\fi}
}
\end{center}
\vspace{-7mm}
\begin{center}
\foreach\x in{1,...,\arabic{num_authors}}{
\textit{$^{\text{\x}}$\checkauthorOrganisation(\x)\cachedata}\ifnum\x<\arabic{num_authors}{,}\fi\\
}
\end{center}
\vspace{-5mm}
\begin{center}
\foreach\x in{1,...,\arabic{num_authors}}{
\checkauthorMail(\x)\def\mystring{\cachedata}\StrLen{\mystring}[\mystringlenn]\ifnum\mystringlenn>0$^{\text{*}}$e-mail:\,\textit{\cachedata}\fi
}
\end{center}
}

\pagestyle{fancy}
\fancyhf{}
\chead{
II Весенняя научная сессия СНО НИЯУ МИФИ -- 2024
}
\cfoot{\thepage}

%%\makeInf{название}{тект аннотации}{Ключевые слова}{Тезисы}
\newcommand{\makeInf}[4]{
\renewcommand{\abstractname}{\normalsize Аннотация}
%%%%%%%%%%%%%%%%%%%%%%%%%%%%%%%%%%%%%%
\begin{center}
\bfseries\large{#1}
\end{center}
%%%%%%%%%%%%%%%%%%%%%%%%%%%%%%%%%%%%%
\vspace{-5mm}
\makeAuthorList
%%%%%%%%%%%%%%%%%%%%%%%%%%%%%%%%%%%%%%
\begin{abstract}
\par
#2
%%%%%%%%%%%%%%%%%%%%%%%%%%%%%%%%%%%%%%%%
\vspace{8mm}
\par
#3
\end{abstract}

%%%%%%%%%%%%%%%%%%%%%%%%%%%%%%%%%%%%%%%%%%%%
\vspace{6mm}
\par
#4
}

\newcommand{\Picture}[3]{%
\begin{figure}[H]
\centering
  \includegraphics[keepaspectratio, width=#3\textwidth]{Pictures/#1}
\caption{#2}\label{#1}
\end{figure}
}

\makeatletter
\renewenvironment{thebibliography}[1]
{ %\refname
	\smallskip
	\nopagebreak
	\centerline{\\ \textbf{\refname}}
	\nopagebreak
	\smallskip
	\@afterheading
	
	%\@mkboth{\MakeUppercase\refname}{\MakeUppercase\refname}%
	\list{
		\@biblabel{\@arabic\c@enumiv}
	}%
	{
		\settowidth\labelwidth{\@biblabel{#1}}%
		\leftmargin\labelwidth
		\advance\leftmargin\labelsep
		\@openbib@code
		\usecounter{enumiv}%
		\let\p@enumiv\@empty
		\renewcommand\theenumiv{\@arabic\c@enumiv}}%
	\sloppy
	\clubpenalty4000
	\@clubpenalty \clubpenalty
	\widowpenalty4000%
	\sfcode`\.\@m}
{\def\@noitemerr
	{\@latex@warning{Empty `thebibliography' environment}}%
	\endlist}
\makeatother

\makeatletter
\renewcommand\@biblabel[1]{#1.}
\makeatother

\newcommand{\p}{\partial}

\begin{document}

\addAuthor{Донецков Андрей Дмитриевич}{НИЯУ МИФИ}{andrey.donetskov@gmail.com}
\addAuthor{Жмелев Глеб Евгеньевич}{НИЯУ МИФИ}{}
\addAuthor{Бакакин Валерий Дмитриевич}{НИЯУ МИФИ}{}

\makeInf
{Методы машинного обучения для анализа переменных звёзд}
{Представлен каскадный pipeline для автоматического выделения и типизации переменных звёзд. 
На первом этапе случайный лес решает бинарную задачу \textit{variable/non–variable}. 
Далее нейронная сеть классифицирует объекты по типам (цефеиды, RR~Лиры, затменные системы).}
{
\begin{table}[h!]
	\caption{Случайный лес для логистической регрессии}
	\centering
	\begin{tabular}{ | l | l | l | }
		\hline
		Столбец 1 & Столбец 2 & Столбец 3 \\
		\hline
		Значение & 16754 & 815\\
		\hline
		Значение & 391 & 1536 \\
		\hline
	\end{tabular}
	\label{table:IvanovII-table1}
\end{table}

\Picture{IvanovII-plot1}{Поверхностный потенциал НЧ FeAu до и после покрытия карбоксиметилдекстраном.}{0.6}%Формат изображения .jpg, .png или .tiff., расширение не указывается

\begin{table}[h!]
	\caption{Пример таблицы}
	\centering
	\begin{tabular}{ | l | l | l | }
		\hline
		Столбец 1 & Столбец 2 & Столбец 3 \\
		\hline
		Значение & Значение & Значение\\
		\hline
		Значение & Значение & Значение \\
		\hline
	\end{tabular}
	\label{table:IvanovII-table1}
\end{table}

Ссылка на рисунок \ref{IvanovII-plot1} или на таблицу \ref{table:IvanovII-table1} оформляется без скобок, метка рисунка соответствует имени файла.

Ссылки на источники желательно оформлять через \texttt{cite}, но также допускается работать с источниками вручную, ссылаться как [1]. Обращаем внимание, что как минимум один из источников должен быть за последние пять лет.

\begin{thebibliography}{9}
	%Согласно ГОСТ Р 7.0.100-2018.
	\bibitem{bibl:IvanovII-vibro}\textit{Бойко В.В., Лапкис А.А.} Построение эталонных виброакустических портретов операций перегрузки
	ядерного топлива [Конференция] // СОТБ: материалы VI Всероссийской конференции и школы для молодых ученых
	– Ростов-на-Дону; - Таганрог: издательство Южного федерального университета, 2019. – стр. 213. 
	\bibitem{bibl:IvanovII-hypmat}\textit{Drachev V.P., Podolskiy V.A., Kildishev A.V.} Hyperbolic metamaterials: new physics behind
	a classical problem [Journal]// Optics Express. V. 21(12). - 2019. - p. 15048.
\end{thebibliography}
}
\end{document}
