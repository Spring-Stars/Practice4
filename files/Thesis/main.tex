\documentclass[oneside, a5paper,10pt]{article}
\usepackage{System/core}

\newcommand*\rfrac[2]{{}^{#1}\!/_{#2}}
\renewcommand{\proofname}{Доказательство}

\newcounter{num_authors}
\setcounter{num_authors}{0}

\newarray\authorName
\newarray\authorOrganisation
\newarray\authorMail

%%\isequal{что сравнить}{с чем сравнить}{если да}{если нет}
%%делает проверку, одинаковы ли макросы "что сравнить" и "с чем сравнить". Если одинаковы - выполняет "если ла", в противном случае - "если нет"
%Команда бесполезна.
\newcommand{\isequalmacros}[4]{
\def\a{#1}\def\b{#2}\ifx\a\b#3\else#4\fi
}

\def\ifEqStringBase#1#2%
{\def\csa{#1}\def\csb{#2}\ifx\csa\csb}

%\ifEqString{СТРОКА1}{СТОКА2}{УС_ДА}{УС_НЕТ}
%сравнивает строки СТРОКА1 и СТРОКА 2. Если равны, то выполняет УС_ДА, если нет -- УС_НЕТ.
\newcommand{\ifEqString}[4]{\ignorespaces
\ifEqStringBase{#1}{#2}#3\else#4\fi
}


%%\addAuthor{ФИО}{организация}{e-mail}
%%Добавляет автора
%%Нумерация массива - С ЕДИНИЦЫ!!
\newcommand{\addAuthor}[3]{
\addtocounter{num_authors}{1}
\authorName(\arabic{num_authors})={#1}
\authorOrganisation(\arabic{num_authors})={#2}
\authorMail(\arabic{num_authors})={#3}
}

%\roflll{\x}
\newcommand{\roflll}[1]{
\ignorespaces
\checkauthorMail(#1)\def\mystring{\cachedata}\StrLen{\mystring}[\mystringlenn]\ifnum\mystringlenn>0\checkauthorName(#1)\cachedata$^{*\text{#1}}$\else\checkauthorName(#1)\cachedata$^{\text{#1}}$\fi}

\newcommand{\makeAuthorList}{
\ignorespaces
\begin{center}
\ifthenelse{\arabic{num_authors} = 0}{}{
\foreach\x in{1,...,\arabic{num_authors}}{
\roflll{\x}\ifnum\x<\arabic{num_authors}{,}\fi}
}
\end{center}
\vspace{-7mm}
\begin{center}
\foreach\x in{1,...,\arabic{num_authors}}{
\textit{$^{\text{\x}}$\checkauthorOrganisation(\x)\cachedata}\ifnum\x<\arabic{num_authors}{,}\fi\\
}
\end{center}
\vspace{-5mm}
\begin{center}
\foreach\x in{1,...,\arabic{num_authors}}{
\checkauthorMail(\x)\def\mystring{\cachedata}\StrLen{\mystring}[\mystringlenn]\ifnum\mystringlenn>0$^{\text{*}}$e-mail:\,\textit{\cachedata}\fi
}
\end{center}
}

\pagestyle{fancy}
\fancyhf{}
\chead{
II Весенняя научная сессия СНО НИЯУ МИФИ -- 2024
}
\cfoot{\thepage}

%%\makeInf{название}{тект аннотации}{Ключевые слова}{Тезисы}
\newcommand{\makeInf}[4]{
\renewcommand{\abstractname}{\normalsize Аннотация}
%%%%%%%%%%%%%%%%%%%%%%%%%%%%%%%%%%%%%%
\begin{center}
\bfseries\large{#1}
\end{center}
%%%%%%%%%%%%%%%%%%%%%%%%%%%%%%%%%%%%%
\vspace{-5mm}
\makeAuthorList
%%%%%%%%%%%%%%%%%%%%%%%%%%%%%%%%%%%%%%
\begin{abstract}
\par
#2
%%%%%%%%%%%%%%%%%%%%%%%%%%%%%%%%%%%%%%%%
\vspace{8mm}
\par
#3
\end{abstract}

%%%%%%%%%%%%%%%%%%%%%%%%%%%%%%%%%%%%%%%%%%%%
\vspace{6mm}
\par
#4
}

\newcommand{\Picture}[3]{%
\begin{figure}[H]
\centering
  \includegraphics[keepaspectratio, width=#3\textwidth]{Pictures/#1}
\caption{#2}\label{#1}
\end{figure}
}

\makeatletter
\renewenvironment{thebibliography}[1]
{ %\refname
	\smallskip
	\nopagebreak
	\centerline{\\ \textbf{\refname}}
	\nopagebreak
	\smallskip
	\@afterheading
	
	%\@mkboth{\MakeUppercase\refname}{\MakeUppercase\refname}%
	\list{
		\@biblabel{\@arabic\c@enumiv}
	}%
	{
		\settowidth\labelwidth{\@biblabel{#1}}%
		\leftmargin\labelwidth
		\advance\leftmargin\labelsep
		\@openbib@code
		\usecounter{enumiv}%
		\let\p@enumiv\@empty
		\renewcommand\theenumiv{\@arabic\c@enumiv}}%
	\sloppy
	\clubpenalty4000
	\@clubpenalty \clubpenalty
	\widowpenalty4000%
	\sfcode`\.\@m}
{\def\@noitemerr
	{\@latex@warning{Empty `thebibliography' environment}}%
	\endlist}
\makeatother

\makeatletter
\renewcommand\@biblabel[1]{#1.}
\makeatother

\newcommand{\p}{\partial}

\begin{document}

\addAuthor{Бакакин Валерий Дмитриевич}{НИЯУ МИФИ}{val-bakakin@yandex.ru}
\addAuthor{Жмелев Глеб Евгеньевич}{НИЯУ МИФИ}{}
\addAuthor{Донецков Андрей Дмитриевич}{НИЯУ МИФИ}{}

\makeInf
{Методы машинного обучения для анализа переменных звёзд}
{Представлен каскадный pipeline для автоматического выделения и типизации переменных звёзд. 
На первом этапе случайный лес решает бинарную задачу \textit{variable/non–variable} (accuracy 93.8\%). 
Далее нейронная сеть классифицирует объекты по типам с общей точностью 95\%. Полученные результаты превосходят аналогичные работы~\cite{bibl:DonetskovAD-1}.}
{
\section{Результаты экспериментов}
\subsection{Бинарная классификация}
Сравнение двух моделей показало преимущество случайного леса (табл. 1), что согласуется с исследованиями~\cite{bibl:DonetskovAD-2}. Высокий recall (79.7\%) гарантирует минимальные потери переменных звёзд на этапе фильтрации.

\begin{center}
  \captionof{table}{Сравнение алгоритмов (F1-score)}
  \begin{tabular}{ l | c | c | c | c }
    Модель & Accuracy & Precision & Recall & F1 \\
    \hline
    Градиентный бустинг & 0.915 & 0.546 & 0.831 & 0.659 \\
    Случайный лес & 0.938 & 0.653 & 0.797 & 0.718 \\
  \end{tabular}
  \label{table:comparison}
\end{center}

\subsection{Мультиклассовая классификация}
Нейронная сеть демонстрирует высокую точность (95\%), но выявлены проблемы с редкими классами:
\begin{itemize}
  \item Низкий precision для затменных систем (12\%) из-за малого размера выборки
  \item Максимальная точность для ротационных переменных (F1 = 0.96)
\end{itemize}

\Picture{DonetskovAD-plot1}{Нормированная матрица конфузий для 6 классов}{0.75}

\section{Заключение}
Разработанный каскадный подход позволяет:
\begin{itemize}
  \item Эффективно фильтровать переменные звёзды (F1 0.718)
  \item Классифицировать основные типы с точностью до 97\% 
  \item Интегрировать данные из разнородных источников
\end{itemize}

Перспективы работы: применение трансформеров для анализа временных рядов и учёт астрофизических ограничений в архитектуре нейросетей.

\begin{thebibliography}{9}
  \bibitem{bibl:DonetskovAD-1}\textit{Johnston K.B. et al.} Variable star classification using multiview metric learning // MNRAS. 2020. V.491(3). P.3805–3816. DOI:10.1093/mnras/staa3221
  \bibitem{bibl:DonetskovAD-2}\textit{Riess A.G. et al.} A 2.4\% Determination of the Hubble Constant // ApJ. 2016. V.826(1). P.56. DOI:10.3847/0004-637X/826/1/56
\end{thebibliography}
}
\end{document}